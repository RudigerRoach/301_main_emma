\documentclass[10pt,a4paper]{article}
\usepackage[utf8]{inputenc}
\usepackage[english]{babel}
\usepackage{amsmath}
\usepackage{amsfonts}
\usepackage{amssymb}
\usepackage{graphicx}

\title{COS 301 Software Documentation}
\author{Melany Barnes 12030466 \\
		Dieter Doman 11002566 \\
		Johan Esterhuyse 10043285 \\
		Rudiger Roach 11004322 }
\date{}
\begin{document}
\maketitle
\begin{center}
Version 1.0 \\
GitHub link: https://github.com/RudigerRoach/301\textunderscore main\textunderscore emma.git \\
\vspace*{5\baselineskip}
\includegraphics[scale=0.5]{Pictures/Logo.png}
\end{center}
\pagebreak
\tableofcontents
\pagebreak
\section{Vision and Scope}

\subsection{Vision}
Our client creates software for camera club event management. A big part of an event comprises of an image judging process. Currently the process is completed by using Infra-red remotes and receivers, but this configuration is limited in terms of usability and the amount of judges that can judge concurrently.\\\\
The proposed solution will replace the hardware remote with a software application to run on a mobile device. The mobile application should alleviate all of the issues caused by the current setup, and should be developed with a server component that plugs into the existing EMMA system.  

\subsection{Scope}
Create a software solution that:
\begin{itemize}
\item Runs on IOS and Android mobile devices.
\item Allows as many as 20+ judges on the night.
\item Allows judges to register against the event (in order to score) by capturing an email address.
\item Remembers the scoring device for future meetings such that registration is not required again.
\item Caters for realtime scoring.
\item Can display a thumbnail image of that currently being judged.
\item Caters for simple score entry bound within a variable range, as well as pluggable scoring methods that could include boolean scoring.
\item Reports meaningful error messages, in a clear way.
\item Allows for quick correction and re-capture.
\item Can notify a judge of outstanding scores.
\end{itemize}

\begin{center}
\advance\leftskip-2cm
\includegraphics[width=160mm]{Pictures/HighLevelUseCase.jpg} 
High Level Use Case Diagram
\end{center}

\pagebreak
\section{Architecture requirements}
\subsection{Architecture requirements}
\subsubsection{Architectural scope}
\begin{itemize}
\item Provide an infrastructure for a judge to rate photos on a mobile device.
\item Provide a database to link a judge's phone id to his email address.
\end{itemize}
\subsubsection{Quality requirements}
\begin{itemize}
\item Usability \\
99 \% of users should be able to use the system with little to no prior training.
\item Scalability \\
The deployed system must be able to operate effectively under the load of 50 concurrent users. The 50 users will be handled by Jetty which creates a thread for each person that connects to certain servlet at a point in time thus it will be able to handle the concurrency. It will be tested by creating a thread pool of 50 threads and doing 50 unit tests concurrently. If further scalability is needed it could be achieved since jetty supports clustering.
\item Installability \\
It should be easy to install the server side component and the effort to get it running each club night should be minimal. There will be a computer at the event running the EMMA server component and our server should be installed on it. The application should also be easy to download and install on the judges phone.
\item Performance requirements \\
 All operations on application should respond within less than 1 second.
\item Testability \\
All services offered must be accompanied by unit tests. The tests should ensure that all pre-conditions are met before the service is delivered and that all post-conditions are met after the service has been delivered.
\item Security \\
The systems functionality should be only available to users who can be authenticated through the EMMA system. The users email address will only be used with no password. New users have to create an account before being granted access to the application if the sessions is closed. If the session is opened any person should be allowed to use it.
\end{itemize}
\subsubsection{Integration and access channel requirements}
\begin{itemize}
\item Integration requirement \\
The production version of this application will need to integrate with EMMA. EMMA is Java a based application.
\item Access channels \\
The mobile application will have to go through a web-service which will be the public interface for the server-side component. It uses HTTP communication between then mobile and the server Jetty component.
\end{itemize}
\subsubsection{Architectural constraints}
\begin{itemize}
\item The mobile application should run on Android and iOS operating systems.
\item The PC's that will be running the server side of the application and EMMA component will generally not be the latest technology(limited memory and processing power).
\item There will be limited to no internet connection.
\item The communication between the mobile device and server PC will be done over a wifi network.
\item The server side component of this project should be able to run on Windows and OS X operating systems.
\end{itemize}
\subsection{Use of reference architectures and frameworks}
\begin{itemize}
\item JIRA Framework for the SCRUM agile method.
\item Appcelerator Titanium framework which is an open-source software development kit for cross-platform mobile development.
\item Jetty for hosting server and servlet handling.
\item TiJasmine framework to run javascript unit  testing in the Titanium framework.
\item JUnit for java unit testing framework.
\end{itemize}
\subsection{Technologies and languages}
\begin{itemize}
\item Java
\item JavaScript
<<<<<<< HEAD
\item XML
\item db40 Object database
\item JavaFX
=======
\item db40 Object database
\item JavaFX
\item FXML
>>>>>>> FETCH_HEAD
\end{itemize}

\pagebreak
\section{Functional requirements and application design}
\subsection{Introduction}
This section discusses the functional requirements for the mobile judging system.
\subsection{Required Functionality}
\subsubsection{Login and Auto Login}
To login for the first time a user will have to enter his email address. The email provided will be authenticated by EMMA. If login fails the user will be informed that he is not registered to be a judge for the current session. If login is successful the device's unique identifier will be sent to the server to be stored in the database so that the device can be remembered on the system. This will allow for auto login - if a user attends a session where he is able to judge his phone will automatically be logged into the system when he enters the application. The user will then be able to use the rest of the system.

\begin{center}
\advance\leftskip-1.3cm
\includegraphics[width=160mm]{Pictures/Login.jpg} 
Login Use Case Diagram 
\end{center}

\subsubsection{Create Judging Session}
When the server is started, it will request that the session's photos as well as all the competition constraints be sent to it. The constraints will contain the type of session (Open event, Closed event, Yes/No, Winner), the range for a valid score and if comments are enabled.

\begin{center}
\advance\leftskip-1.3cm
\includegraphics[width=130mm]{Pictures/CreateJudgingSession.jpg} 
Create Judging Session Use Case Diagram 
\end{center}

\subsubsection{Voting}
If a user logs in before the event starts, a loading screen will be displayed until the event starts. The server will inform the user's application when the event starts. The server will then pass through the first image and the details about the image. The details will contain the image name, the bottom and top score ranges as well as if comments must be enabled. The users will vote for the image and will be able to leave a comment if the comments are enabled. The server will either have a time limit per image or the server will check if all users have submitted their vote. If not all users have submitted their vote, the server will notify those users. If all users have submitted their vote, the next image and its details will be displayed. This will continue until all images have been scored. The user will be notified that voting is done.
\\\\
\textbf{Number based scoring\\}
The server sends the image name, current image path, the bottom and top score ranges as well as if comments must be enabled  to the application. The user can choose to move the slider to change the score or he can click on the score text area to change the score by typing in the score on the on-screen keyboard. When the user is satisfied with the score he gave, he should click on the submit button. The server will then receive the score and the following image will be send to the application when ready or the user will be notified that the session is complete. If the session is complete the user will be logged out and taken to the login screen.
\\\\
\textbf{Elimination round\\}
The server sends the image name, current image path as well as if comments must be enabled to the application. The user can click on the Yes or No button. When the user is satisfied with his choice, he should click on the submit button. The server will then receive his choice and the following image will be send to the application when ready or the user will be notified that the session is complete. If the session is complete the user will be logged out and taken to the login screen.
\\\\
\textbf{Choose a winner\\}
The server sends a list of image names as well as an array containing all the image paths to the application. The user can swipe through the images on his own time and select any image as the winner. When the user is satisfied with his choice, he should click on the submit button. The server will then receive his choice and the following image will be send to the application when ready or the user will be notified that the session is complete. If the session is complete the user will be logged out and taken to the login screen.
\\  
\begin{center}
\advance\leftskip-1.cm
\includegraphics[width=80mm]{Pictures/Vote.jpg} \\
Voting Use Case Diagram 
\end{center}

\subsubsection{Internationalisation}
The application is able to handle different languages. The languages are device dependant, thus the application text will be displayed in the language that the device is currently set to. The default language is English, thus if the device is set to an unsupported language the application will default to English. The following languages are supported: English, Afrikaans, French and Dutch.

\subsection{Use case prioritization}
Critical Use Cases are the main cases that the system is made up of namely: Login, Create Judging Session and Voting. Without these cases the system will have limited to no functionality which will lead to a system that is not required by anyone.
\\ \\
Important Use Cases are the cases that improves the critical use cases and introduces a wider variety of functionality. These cases are Auto Login.

%Nice-To-Have Use Cases make the system more user friendly and provides a better user experience. This case is: 

\subsection{Use case/Services contracts}
\begin{center}
\advance\leftskip-1.3cm
\includegraphics[width=160mm]{Pictures/servicesContractLogin.jpg} 
Services contract for Login
\end{center}

\begin{center}
\advance\leftskip-1.3cm
\includegraphics[width=160mm]{Pictures/servicesContractCreateJudgingSession.jpg} 
Services contract for Create Judging Session
\end{center}

\begin{center}
\advance\leftskip-1.3cm
\includegraphics[width=160mm]{Pictures/servicesContractVote.jpg} 
Services contract for Voting
\end{center}

\subsection{Process specifications}
\begin{center}
\includegraphics[width=120mm]{Pictures/LoginActivityDiagram.jpg} 
Login Activity Diagram 
\end{center}

\begin{center}
\advance\leftskip-2cm
\includegraphics[width=170mm]{Pictures/createJudgingSessionActivityDiagram.jpg} 
Create Judging Session Activity Diagram 
\end{center}

\begin{center}
\advance\leftskip-2cm
\includegraphics[width=170mm]{Pictures/VotingActivityDiagram.jpg} 
Voting Activity Diagram 
\end{center}

\subsection{Domain objects}
\section{Testing Information}
\subsection{Unit testing}
\paragraph{}For unit testing we used the Jasmine framework for the mobile side of our system which is a javascript unit testing framework. We used an mock java server for mobile unit testing and mock data to test specific functions in the code.
\paragraph{} For the server side component of our system we used a JUnit testing framework. Using HTTP default clients to perform mock network calls for the tests and mock data to test certain components of the server.
\paragraph{}Mock data was set up to test the pre- and post conditions of the software contract.
\subsection{Usability testing}
\paragraph{}Our mobile side was developed focusing on user centered development. Throughout the design and implementing of the mobile side users where consulted in short feedback cycles, getting there inputs and new ideas on how to make the application as usable as possible.
\paragraph{}
During our test we had a user group of 6 random people. We run the 3 different types of sessions with them. We gave them some tasks to do such as rotating the screen, change the language, and to evaluate the flow of voting. 
\paragraph{} 
\advance\leftskip-1.3cm
\includegraphics[scale=0.55]{Pictures/ConsentForm.pdf}
\pagebreak
\paragraph{} Task that needed to be performed
\begin{enumerate}
\item Change the server address in the settings tab.
\item Change the comment setting in the settings tab.
\item Login into the session.
\item Submit one score using a textbox.
\item Submit one score using a slidebar.
\item Finish the session.
\item For an elimination session, push yes then no then submit.
\item Scroll between photos to select a winner.
\item Add an comment and submit your winner.
\end{enumerate}
\paragraph{}
\includegraphics[scale=0.55]{Pictures/results.jpg}
\subsection{How tests where executed}
\paragraph{}
Junit netbeans plugin for the server side.
TiJasmine modules and manual mock-server execution for the mobile side.   
\subsection{Test coverage}
\begin{itemize}
\item Approximately 60 percent test coverage on the mobile side.
\item Approximately 70 percent test coverage on the server side. 
\end{itemize}

\section{Project management}
\subsection{Software development process}
We followed the scrum software development methodology. Its agility, in terms of requirements, meant that the project was able to grow during development, while we where discovering new requirements from the client.
\subsection{Issue management}
On the clients suggestion, we used the Jira issue tracking system, combined with git and gitHub for version control.
\pagebreak
\subsection{Team profile}
Melany Barnes:
\begin{itemize}
\item Team Lead.
\item Developing the bulk of the mobile app front-end.
\end{itemize}
Dieter Doman:
\begin{itemize}
\item Developing the back-end of the server app.
\end{itemize}
Johan Esterhuyse:
\begin{itemize}
\item Developing the front end for the server app.
\item Paying special attention to documentation and diagrams.
\end{itemize}
Rudiger Roach:
\begin{itemize}
\item Developing the back-end of the mobile app.
\item Developing the database for the server app.
\item Assisted in development of the mobile app front-end.
\end{itemize}
\subsection{Project progress}
We aimed to provide burndown charts for the entire project progress but unfortunately The issue tracking system ran from a server at the clients' residence and it experienced a lot of down-time. Luckily we can provide burndown charts for the following sprints:
\begin{itemize}
\item Authentication\\
\includegraphics[width=80mm]{Pictures/Auth.png}
\item Auto login\\
\includegraphics[width=80mm]{Pictures/ALchart.png}.
\item Voting with 1 image\\
\includegraphics[width=80mm]{Pictures/v1i.png}
\end{itemize}
\subsection{Un-implemented functionality}
\begin{itemize}
\item Displaying results after voting has completed after.
\item The mobile app is unaware of a server crash and will wait indefinitely in this case.
\end{itemize}
\subsection{Main risks and challenges faced}
\begin{itemize}
\item Getting to know Jira and issue tracking.
\item Getting to know the Appcelerator platform.
\item Getting to know JavaFX using FXML.
\item Multi-platform support.
\item Multi-screen-resolution support.
\end{itemize}
\section{User manual for server}
\subsection{Creating a session}
\includegraphics[scale=0.5]{Pictures/serverMain.jpg}
\begin{enumerate}
\item Specify the Session type:
\begin{enumerate}
\item Normal is a session where the user rates each picture based on general impression.
\item Yes/No is a session where the user states whether they think a picture should go to the next round or not.
\item Winner is a session where the user chooses one winner and the picture with the most votes wins.
\end{enumerate}
\item Specify if the session is a controlled session or not.
\begin{enumerate}
\item If the creator of the session would like to guide the judges through voting then tick the box.
\item If the creator of the session would like judges to judge on their own time then don't tick the box.
\end{enumerate}
\item Specify if the session should be open to all judges.
\begin{enumerate}
\item If anyone can judge then tick the box.
\item If only specific people may judge then don't tick the box.
\end{enumerate}
\item Specify whether comments should be enabled.
\begin{enumerate}
\item If you want the judges to give their opinion then tick the box.
\item If you don't want judges to give their opinion then don't tick the box.
\end{enumerate}
\item Specify the minimum and maximum scores.
\begin{enumerate}
\item Minimum has to be smaller than maximum.
\end{enumerate}
\item Add Images to the voting session.
\begin{enumerate}
\item Click the "+" button to add a single image.
\begin{enumerate}
\item Click browse and select an image.
\item Type in a description.
\end{enumerate}
\item Click the "++" button to import multiple images.
\begin{enumerate}
\item Click the browse next to the "select images" and select the images.
\item Click the browse next to the "select description file" and select the description file.
\begin{enumerate}
\item The description file is a file which specifies the details for each image one line at a time.
\end{enumerate}
\end{enumerate}
\item Click the gear button to edit a selected image.
\begin{enumerate}
\item First select the image.
\item Click the gear.
\item Specify the changes.
\end{enumerate}
\item Click the "-" button to delete a selected image.
\begin{enumerate}
\item First select the image.
\item Click the "-" button to remove the selected image.
\end{enumerate}
\end{enumerate}
\item Add Judges to the voting session(only neccesary if open judging(see 3. of manual) is not enabled).
\begin{enumerate}
\item Click the "add" button to add a judge.
\begin{enumerate}
\item Specify the judges email address and click add.
\end{enumerate}
\item Click the "delete" button to remove the selected judge.
\begin{enumerate}
\item First select the judge.
\item Click the "-" button to remove the selected judge.
\end{enumerate}
\item Click the "import" button to import judges.
\begin{enumerate}
\item Click browse and select a judges file.
\begin{enumerate}
\item A judges file is a file that specifies the email addresses of judges one at a time.
\end{enumerate}
\end{enumerate}
\item Click the "export" button to export images.
\begin{enumerate}
\item Browse to where you want to save the file and click save.
\end{enumerate}
\end{enumerate}
\item Click "Start Session" in order to start running a session.
\begin{enumerate}
\item If any data is missing, the creator will be informed.
\end{enumerate}
\end{enumerate}
\subsection{Controlling the session:}
\includegraphics[scale=0.5]{Pictures/serverRunning.jpg}
\begin{enumerate}
\item Click the "Next" button in order to preview the next image.
\begin{enumerate}
\item If this is a controlled session(See session creation point 2), this will take the judges(using app) to the next image.
\end{enumerate}
\item Click the "Previous" button to go back to the previous image.
\begin{enumerate}
\item This is not possible if the session is controlled(See session creation point 2).
\end{enumerate}
\item When all judges are done voting click on "Finalize Session".
\begin{enumerate}
\item This will end the judging and then display the results.
\end{enumerate}
\end{enumerate}
\subsection{Importing project for further development}
Since by the client's request the only wanted an netbeans project. Thus open project in netbeans after pulling from our git repository and import the all the external libraries under external libraries folder.
\subsection{Installing the server}
\paragraph{}
Run following command after building project in netbeans(which will create jar file in dist folder) in command line in the dist folder: "java -jar YouRate Simulator.jar"
\section{User manual for mobile application}
\paragraph{}
When the application opens up the app will first try to automatically log in (1.1). If the user has logged in using the device before, the server will recognise the device and automatically log the user into the app. If automatic login fails, the user will be redirected to the login screen.
\paragraph{}
Waiting screens with different messages might be encountered:
\begin{itemize}
\item Attempting automatic login… (See 1.1)
\item Attempting to log in… (See 1.2)
\item Waiting for server…  (See 1.3)
\end{itemize}
\paragraph{}
The email address is validated, thus a valid email address must be entered. A message will be displayed if an invalid email address was entered and the login button was clicked. If the email address is valid the app will try to log in and a waiting screen (1.2) will be displayed. When the user is logged in successfully, waiting screen (1.3) will be shown. When an image on the server is ready to be displayed, the application will continue to the correct type of voting page (1.4, 1.5, 1.6)
\paragraph{}
Number based scoring (1.4):
\paragraph{}
The score can be changed by moving the slider or by clicking on the score text area. When a user clicks on the text area, the device onscreen keyboard will pop up and the user can enter a score.
\paragraph{}
\includegraphics[scale=0.3]{Pictures/ScoreKeyboard.png}
\paragraph{}
\includegraphics[scale=0.3]{Pictures/ScoreSlider.png}
\paragraph{}
Elimination round (1.5): 
The user can choose between yes or no per image and then click on the submit button to submit their result.
\paragraph{}
\includegraphics[scale=0.3]{Pictures/YesNo.png}
\paragraph{}
Choose a winner (1.6):
The user can swipe through the images and choose 1 as the winner and then click on the submit button to submit their result.
\paragraph{}
Changing the settings:
The settings can be changed by clicking the device hardware menu button, if the device does not have a hardware menu button a settings icon will be displayed on the screen. The settings page will look as follows:
\paragraph{}
\includegraphics[scale=0.3]{Pictures/Settings.png}
\paragraph{}
The server address can be changed by clicking on “Set Server Address”. To successfully connect to the server, the correct server address must be entered. Example address: 192.168.0.100 or www.uRate.com
To change the comment entry method, the “Set comment entry method” can be clicked. Comments can be entered via clicking a button or by having a textbox always visible on the screen.
\section{Glossary}
EMMA - Entry and Member Management Application\\
His - Refers to his/her\\
He - Refers to he/she
\end{document}